\documentclass[14pt]{beamer}
% Možne velikosti pisav so: 8pt, 9pt, 10pt, 11pt, 12pt, 14pt, 17pt, 20pt

% Nastavitve za šumnike
\usepackage[T1]{fontenc}
\usepackage[utf8]{inputenc}
\usepackage[slovene]{babel}

% To potrebujemo, da bomo lahko delali zapiske za predavatelja
\usepackage{pgfpages}

% Treba je ločiti zapiske za predavatelja in prosojnice za občinstvo.
% To lahko naredimo v beamerju, glej
% https://gist.github.com/andrejbauer/ac361549ac2186be0cdb

% Primer tako narejenih prosojnic:
% http://math.andrej.com/wp-content/uploads/2016/06/hott-reals-cca2016.pdf

% Izberemo, ali bomo naredili samo prosojnice, samo zapiske ali oboje

%\setbeameroption{hide notes}                        % samo prosojnice
%\setbeameroption{show only notes}                   % samo zapiski
\setbeameroption{show notes on second screen=right}  % oboje

% Ali naj se \item-i pojavljajo zaporedoma, ali vsi naenkrat?
% Naj se pojavljajo zaporedoma:
% \beamerdefaultoverlayspecification{<+->}

% Minimalistični still
\mode<presentation>
% \usetheme{Goettingen}
% \usecolortheme{rose}

% Malo manj dolgočasna pisava
\usepackage{palatino}
\usefonttheme{serif}

% Izklopimo navigacijske simbole, ker so neuporabni
\setbeamertemplate{navigation symbols}{}

% Aktiviramo oštevilčenje strani
\setbeamertemplate{footline}[frame number]{}

% Stil za zapiske
\setbeamertemplate{note page}{\pagecolor{yellow!5}\insertnote}

\usepackage{amsmath}
\usepackage{amssymb}
\newtheorem{izrek}{Izrek}

\begin{document}

\title{Kako pripravimo prosojnice}
\author{Andrej Bauer}
\date{Planet Zemlja}

\begin{frame}
  \titlepage

  \note[item]{Povej dober vic, da te začnejo poslušati.}
\end{frame}

\begin{frame}
  \frametitle{Zakaj imajo prosojnice naslove?}

  \begin{itemize}
  \item Da predavatelj ve kaj se dogaja.
  \item Ker se predavatelju ne da razmišljati.
  \item Ker je Powerpoint tako naprogramiran.
  \item Tako naštevanje je dolgočasno.
  \end{itemize}


  \note[item]{Tu se delaj norca iz prosojnic z naslovi.}
  \note[item]{Pokritiziraj prosojnice, ki samo nekaj naštevajo.}
  \note[item]{Vprašaj, ali manjka kakšna vejica.}
\end{frame}

\begin{frame}
  \frametitle{Prikažimo vsebino po kosih.}

  \begin{izrek}
    Za vse $x, y \in \mathbb{R}$ velja $x^2 + y^2 \geq 2 x y$.
  \end{izrek}

  \pause

  \begin{proof}
    Vemo, da velja $(x - y)^2 \geq 0$, torej
    %
    \begin{equation*}
      x^2 - 2 x y + y^2 \geq 0.
    \end{equation*}
    %
    \pause
    Ko na obeh straneh prištejemo $2 x y$, dobimo želeno neenačbo
    %
    \begin{equation*}
      x^2 + y^2 \geq 2 x y. \qedhere
    \end{equation*}
  \end{proof}

  \note[item]{Ukaz \texttt{{\char92}pause} vstavi premor v prosojnice.}
  \note[item]{Pojasni, da se ne piše ``željen''.}
\end{frame}

\begin{frame}
  Tu je slika iz interneta:
  %
  \begin{center}
    (Vstavi sliko sem.)
  \end{center}

  \note[item]{Vprašaj, kako se vstavi sliko, da vidimo ali se še spomnijo.}
  \note[item]{Vstavi sliko, pusti študentom, da izberejo vsebino slike, da boš imel boljše študentske ankete.}
  \note[item]{Ne pretiravaj s cinizmom.}
  \note[item]{Vprašaj, v čem je razlika med ironijo in cinizmom.}

\end{frame}



\end{document}
