\documentclass[a4paper]{article}
% documentclass je ukaz
% document je okolje

% Nekateri ukazi in okolja sprejmejo tudi dodatne argumente, ki
% so lahko obvezni: {argument} ali neobvezni: [argument].


% Vključite pakete za podporo slovenščini,
% prepoznavo vhodnega kodiranja `utf8` in izhodnega kodiranja `T1`.
% Paket amsmath za vsak slučaj, če bi te inline enačbe zajebavale.
% Paket vključimo z ukazom \usepackage[options]{package}
\usepackage[utf8]{inputenc}
\usepackage[T1]{fontenc}
\usepackage[slovene]{babel}
\usepackage{lmodern}
\usepackage{hyperref}
\usepackage{amsmath}

% preambula = vse, kar je med documentclass in begin{document}; do tu

\begin{document} % Začetek okolja document, paired with: \end{document}, 
% ki je konec okolja document.

\title{Vaja \LaTeX a}
\author{zmrda}
\date{26.\ 01.\ 2025}
\maketitle 

% Da se bo glava dokumenta s temi podatki prikazala, morate 
% uporabiti ukaz `\maketitle`.


% Nov razdelek:
\section{Ukazi in okolja}

Okolja: abstract, verbatim, itemize, enumerate. 
% Za enumerate sta v HTML znački ul in ol.
Okolja za matematični način: math (za vrstičnega) in displaymath 
[za prikaznega]. 


\subsection{Enačbe v vrstičnem in prikaznem načinu}
% VRSTIČNI NAČIN:  Matematični \( izraz sredi \) vrstice.
\subsubsection{Vrstični način:}
Iskanje ničel kvadratne enačbe: \(x = \frac{-b \pm \sqrt{b^2 - 4ac}}{2a}\)


% PRIKAZNI NAČIN:  \[ diva oz. matematični izraz v svoji vrstici \]
\subsubsection{prikazni način:}
\begin{center}
   \[k_1\cdot u_1+k_2\]
   \[k_1 \cdot u_1 + k_2 \cdot u_2 + \ldots + k_{n-1} \cdot u_{n-1} + k_n \cdot u_n\]

\end{center}% \[ \]



\subsection{Formule idr. uporabni ukazi za zapisovanje enačb}
% Ukaz frac za ulomek, ki sprejme 2 argumenta --> {}{}:
\begin{raggedright}
Ulomek: 
$\frac{22}{7}$ \\
\end{raggedright}
% NE POZABI NA DOLARČKE, KER DOBIŠ DRUGAČE ERROR! (KER JE ENAČBA)
% Dolarčki namreč pomenijo matematično okolje, matematični ukazi delujejo 
% samo znotraj teh: frac, int, ukazi za matrike itd.

Matrika:
\[
\begin{pmatrix}
a & b \\
c & d
\end{pmatrix}
\]
%
%
If you want to use different types of brackets, you can replace pmatrix \\
with one of the following:
\begin{itemize}
    \item matrix for no brackets
    \item bmatrix for square brackets
    \item vmatrix for vertical bars
    \item Bmatrix for curly braces
\end{itemize} 
%
Integral: \\
a simple integral: $\int_a^b f(x) \, dx$ \\
a double integral: $ \iint_D f(x, y) \, dx \, dy$

% Ukaz rule za ločnice, ki nariše pravokotnik (sprejme dva obvezna 
% argumenta in enega neobveznega). [višina]{dolžina}{širina}:
%$$\rule{20}{5}$$


\subsection{Okolja}
Itemize:
\begin{itemize}
    \item item1
    \item item2
    \item item3
\end{itemize}
% <-- procentek zato, da se Enumerate ne zamakne
Enumerate:
\begin{enumerate}
    \item enumerateditem1
    \item enumerateditem2
    \item enumerateditem3
\end{enumerate}





\section{Zgradba dokumenta = razdelek oz. section}
\subsection{Podrazdelek oz. subsection}
\subsubsection{Podpodrazdelek oz. subsubsection}

Line break: double backslash. \\
New line.


New paragraph: leave an empty row above. \\
\begin {raggedright}
\\ Paragraph break: double backslash twice.
\end{raggedright}

% raggedright, da se razreši opozorilo "Underfull \hbox (badness 10000)"








\end{document}